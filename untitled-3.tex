%\documentclass[12pt,journal,draftcls,letterpaper,twocolumn]{article}
\documentclass[conference]{IEEETran}

%\usepackage{graphicx}
\usepackage{amsmath,amssymb}    % need for subequations
\usepackage[dvips]{graphicx}   % need for figures
\usepackage{verbatim}   % useful for program listings
\usepackage{color}      % use if color is used in text
\usepackage{subfigure}  % use for side-by-side figures
\usepackage{hyperref}   % use for hypertext links, including those to external documents and URLs
\usepackage{epsfig}
\usepackage{float}
\usepackage{listings}


\begin{comment}
\pagestyle{empty} % use if page numbers not wanted
\end{comment}
\DeclareMathOperator*{\argmin}{arg\,min}

% above is the preamble

\begin{document}

\title{\textbf{ UV Data Analysis and Navigation}}
\author{\authorblockN{{Aditya Modi, Jerrid Mathews, Mario Gerla } }\authorblockA{Department of Computer Science\\
University of California, Los Angeles \\
Los Angeles, CA 90024. \\
\{adityam, matth122, gerla\}@cs.ucla.edu}}

\date{}
\maketitle


\begin{abstract}
Pedestrian’s exposure to UV Radiation depends on many factors like geographic location and environmental properties. UV radiation in moderation is beneficial to human health. However, overexposure to UV Radiation results in many health risks including skin cancer. The purpose of this project is to route the user via a path, which has minimum UV exposure. Data analysis is done to determine minimum number of UV samples to assert the UV exposure on a path. an accuracy of above 75\% is aimed for UVA and above 95\% is aimed for UVB. A method is also devised to estimate UV data from the neighbors if the number of samples for a given road segment is lesser than the minimum number of UV samples required. UV data is gathered by actual experiments and statistical analysis is performed on the data. A Web Service is built to cater the developers to use the system in order to get the best path to route a pedestrian with minimum UV exposure. A web application is also built to consume the web service.
\end{abstract}

\section{Introduction}
Scientists have come to a similarly dichotomous recognition that exposure to the ultraviolet radiation (UVR) in sunlight has both beneficial and deleterious effects on human health. UV exposure is beneficial to humans in a optimum amount since it is necessary for the production of Vitamin D in the body. Unlike other essential vitamins, which must be obtained from food, vitamin D can be synthesized in the skin through a photosynthetic reaction triggered by exposure to UV radiation \cite{dep1}. So, it is necessary for humans to get an optimum amount of UV exposure. However, over exposure of UV have a lot of bad implications on the body like skin cancer and sun burns. These health effects could have been worsened due to the depletion of the ozone layer resulting in increase in UV radiation \cite{dep2}. The effect on human health of solar UV at the earth’s surface depends on many factors like geographic location and environmental properties.  

The UV spectrum, shown in Figure \ref{fig:uvSpectrum} is divided into three wavelengths: UVA 315-400nm, UVB 280-315nm and UVC 200-280nm \cite{dep4}. The ozone layer of earth blocks UVC radiations where as UVA and UVB radiations pass through the atmosphere. UVA radiation penetrates deeply into the skin, where it can contribute to skin cancer indirectly via generation of DNA-damaging molecules such as hydroxyl and oxygen radicals. Sunburn is caused by too much UVB radiation; this form also leads to direct DNA damage and promotes various skin cancers. Erythema is caused primarily by the UVB \cite{dep3}. It causes reddening of the skin by damaging the epidermal layer. The amount of UV radiation that is absorbed or scattered is determined by a number of factors. For measurement of UV Radiation, there is a metric called UV index \cite{dep5}. 
\begin{figure}
\begin{center}
\includegraphics[scale=0.45]{uvSpectrum.png}
\caption{\small \sl UV Spectrum\label{fig:uvSpectrum}}
\end{center}
\end{figure}
\\

Since over exposure of UV on humans has several adverse effect, exposure to UV radiation should be avoided. The paper highlights a web service to route users through routes with minumum UV exposure and thus reducing the amount of UV exposure on pedestrians. It also showcases a prototype to consume the web service by building a web application. 

In order to estimate the UV levels on a route, analysis was done in order to determine minimum number of samples necessary to assert the UV levels on a particular route. The analysis plays a crucial role in the navigation application since there needs to be a certain confidence in estimating the UV levels on a particular route. Another interesting problem that is discussed in the paper is to estimate the UV on a route when the number of samples are less than the actual number of samples required. The UV level in these cases is estimated from the UV levels of the neighbors. Using the results mentioned in ~\cite{uvguardian}, it is resonable to estimate the UVA accuracy to 75\% and UVB accuracy to 95\%. 


\section{Approach}
Work is mainly divided into the following parts:
\begin{itemize}
\item{Route Selection}
\item{Data Collection}
\item{Data Analysis}
\item{Web Service Development}
\item{Navigation Web Application}
\end{itemize}
For the project to be a success, the most crucial part is selection of the route for the experiment. The route that is selected needs to have a good mix of trees, buildings and open spaces. A major criteria was to also select a path which has alternate routes so that decision on suggesting a route can be taken by comparing the options based on the UV levels. After the route selection, a selection was to be made for selecting the devices to collect the UV data. A requirement for sensor selection was that device should have sensors to measure both UVA and UVB data. Also there needs to be an interface to transfer the data collected to the computer.  A GPS device is also necessary to keep a track of the latitude and longitude points where the UV data is collected.  

After the data was collected, analysis was done on the data to determine the minimum number of samples necessary to assert the level of UV on a particular route. After the analysis is completed, a web service is developed to incorporate the analysis done so far, and help the developers to create platform independent navigation system on top of the system created. The main idea being the users of the application uploading data to the common server and the web service provides an API to the external world developers. Finally, a web application was also created to consume the web service and provide as a basis for other application developers. 

\section{Data Collection}

Data collection had two main components, UV readings and GPS readings. For UV Readings, a sensor built at UCLA was used ~\ref{fig:uvSensor}. Since there was no GPS device built on the UV sensor, Android device was used to collect GPS readings. GPS logger software ~\ref{fig:gpsLogger} was used to log the GPS readings, which were at a later stage merged with the UV sensor readings to get the UV exposure at specific latitude and longitude points. 
\begin{figure}
\begin{center}
\includegraphics[scale=0.55]{uvSensor.png}
\caption{\small \sl Sensor to collect UV Data.\label{fig:uvSensor}}
\end{center}
\end{figure}
\begin{figure}
\begin{center}
\includegraphics[scale=0.55]{gpsLogger.png}
\caption{\small \sl Software to collect GPS Data.\label{fig:gpsLogger}}
\end{center}
\end{figure}

\subsection{Choosing a Route}
Selecting a route was extremely crucial for the project since UV exposure is affected by geographic and environment properties like trees and buildings. A route should be selected with a good balance of open spance, trees, buildings, and a mixture of all previously mentioned objects. After guaging different route options, a route was selected around UCLA area which had a mix of open space, buildings and trees.  Our source was 606 Levering Ave and destination was 11020 Kinross Ave ~\ref{fig:Route}.  The routes were as follows
 \begin{itemize}
\item  Via Veteran Ave
\item Via Levering Ave and Gayley Ave
\item Via Weyburn Ave
 \end{itemize}
\begin{figure}  
\begin{center}  
\includegraphics[scale=0.4]{allRoutes.png}
\caption{\small \sl Routes taken for doing the analysis.\label{fig:Route}}  
\end{center}  
\end{figure} 
\subsection{Data Collection}
UV Readings were taken using the sensor device on the selected route ~\ref{fig:Route}. Readings were taken between 9a.m. and 11a.m. on couple of sunny days with clear sky. Protection from the sun rays may be different on different sides of the road depening on height of the buildings, presense of trees, time of days etc. Thus, readings were taken by walking on both sides of the road wherever possible. Figure ~\ref{fig:dataPoints} shows a plot of points where readings were taken by walking along the road.
\begin{figure}
\begin{center}
\includegraphics[scale=0.4]{dataPoints.png}
\caption{\small \sl Data points with UV Readings.\label{fig:dataPoints}}
\end{center}
\end{figure}


\subsection{Data Cleansing}
Data cleansing is the most important step of any project where data analysis is involved since we need clean data to perform our analysis on. As there were different devices for taking the UV readings and GPS readings, a script was written to merge the readings so that exact UV readings were collected on the logged GPS points. There were two sensors for each UVA and UVB. So the data was analyzed to determine which of the two sensors readings were more reliable. There were also some random erroneous readings reported for the sensors sometimes which were discarded and before the data analysis was done, it was cleaned. 

\subsection{Data Analysis}
The main aim of the project is to suggest the best route to the user out of the many routes options he could take. The alternate routing data is taken from the Google Maps Web Service. In order to determine the various alternate route options that can be taken to travel from source to destination, Google Maps Web Service is called. It returns the various routes that can be taken which are called as “Legs” and the various sections of the roads that changes are called as “Steps”. In the route selected for the experiment, each of the red lines in Figure ~\ref{fig:allRoutes} indicate each leg and each of the red lines between the markers in Figure ~\ref{fig:stepsInALeg} indicate each step. Minimum number of points needed to assert the UV level with a certain confidence was to be found. Analysis was initially done by taking one step and finding the number of points required on that step. However after realizing that the step was too big to perform analysis, step was broken down into multiple segments. In order to get number of segments within a step, Open Street Maps data was used where steps are divided into small segments.  Figure ~\ref{segmentsInAStep} indicates segments within a single step. 

\begin{figure}
\begin{center}
\includegraphics[scale=0.35]{stepsInALeg.png}
\caption{\small \sl Steps in a single Leg.\label{fig:stepsInALeg}}
\end{center}
\end{figure}
\begin{figure}
\begin{center}
\includegraphics[scale=0.35]{segmentsInAStep.png}
\caption{\small \sl Segments in a single Step.\label{fig:segmentsInAStep}}
\end{center}
\end{figure}

Data is analyzed to determine the minimum number of points required to get the accuracy of UVA above 75\% and accuracy of UVB above 95\%. Bootstrapping method is used to determine the minimum number of points. Bootstraping works as follows:
\\
\\
{ 	\boxed{
		\begin{array}{l}
			\text{1. Take the average of all the points on a particular }
		\\ \quad	\text{  segment which would become our actual average }
		\\    \text{2. Select one point randomly}
		\\    \text{3. Calculate the accuracy with respect to the actual}
		\\ \quad	\text{averge}
		\\	\text{4. Select one more point randomly}
		\\	\text{5. Take the avegare of all the points selected till now}
		\\    \text{6. Calculate the accuracy with respect to the actual}
		\\ \quad	\text{ averge}
		\\	\text{7. Repeat from Step 4 till desired accuract is achieved}

		\end{array}
}
\begin{center} 
{\small \sl Algorithm for Boostraping }
\end{center}
}



\definecolor{listinggray}{gray}{0.9}
\definecolor{lbcolor}{rgb}{0.9,0.9,0.9}
\lstset{language=R, tabsize=4, backgroundcolor=\color{lbcolor},      caption=R Code for Bootstraping,
        commentstyle=\color[rgb]{0.133,0.545,0.133},
        stringstyle=\color[rgb]{0.627,0.126,0.941},        breaklines=true,
        frame=single}
\begin{lstlisting}
route <- read.csv("route.txt", header=TRUE  , sep=",")
for(noOfSample in 1:10){  
  trueMean = mean(route$uvb)
  uvb = route$uvb #getUVB Readings
  uvbReadingsVector <- vector()
  for(i in 1:10){
    uvbSample = sample(uvb,noOfSamples)
    sampleMean = mean(uvbSample)
    accuracy =  abs((trueMean-sampleMean)/ sampleMean*100)
  }
}\end{lstlisting}


Bootstrapping method is chosen since the distribution is completely random and the readings along the segments are not consistent. Bootstrapping is the best method in these cases. For analysis variety of segments are covered by considering different road segments at different orientations and different length of segments. 

Detailed analysis of one of the segments that was considered is as follows. A segment from Veteran Avenue was selected. The selected segment had a trees on one side of the road and buildings on other side of the road. Start Point was 34.060067, -118.449498 and End Point was 34.058945, -118.44864. Total length of the segment was 480 ft.\\
\begin{figure}
\begin{center}
\includegraphics[scale=0.3]{veteranMap.png}
\includegraphics[scale=0.3]{veteranSatellite.png}
\caption{\small \sl Sample selected segment map view}\label{fig:veteranSatellite}
\end{center}
\end{figure}

\begin{table}[h]
\centering
\begin{tabular}{|c|c|c|}
\hline
No of Readings & UVA Accuracy & UVB Accuracy \\
\hline 
1 & 72.03039 & 93.37286\\
\hline
2 & 72.84915 & 93.7059 \\
\hline
3 & 73.27921	& 94.36731\\
\hline
4 & 78.09173	& 94.94913\\
\hline
5 & 76.75012 & 95.06996 \\
\hline
6 & 77.60953 & 97.47088 \\
\hline
7 & 78.30137 & 97.91901\\
\hline
\end{tabular}
\caption{\small \sl Accuracy table for bootstraping}
\end{table}

As observed from the table that UVA and UVB take around 4 readings to get the desired accuracy. After repeating this experiment for many different segments, observation was made that in the worst case, 6 points on a segment are necessary to get accuracy above 75\% for UVA and 4 points on a segment are necessary to get accuracy above 95\% for UVB.
In average case, 4 points on a segment are necessary to get accuracy above 75\% for UVA and  3 points on a segment are necessary to get accuracy above 95\% for UVB.


\subsection{Less number of Segments Analysis TODO: change the subsection name}
There may be some segments, which may not have the minimum number of readings that is necessary to assert the UV values like the segment in the Figure ~\ref{fig:lessReadings}. 

The problem of some segments not having desired number of readings is countered in the way mentioned below. In that case, a weighted average of the readings from the neighboring segments is taken. The weight is in terms of how far the segment is from the segment, which is currently under consideration and also, the number of readings that the current segment has. The immediately neighboring segments have a weight of 0.5, the weight is decreased by 0.1 as segments farther from the segment under consideration is taken into account for calculating the UV values. This method was selected since segments, which are near the segment under consideration, have a high correlation of readings with the segment. The window of neighboring segments is expanded till the desired number of readings for the segment is achieved. 

For instance, if the segments in Figure ~\ref{fig:lessReadings} is considered. The segements end points are marked by blue colored markers and the red color markers indicate readings at those points. As observed in the figure, the number of readings are less than the minimum number of readings required. So, a weighted average of the neighboring segments is taken. Since both the segments are immediate neighbors, weight would eventually be proportional to the number of readings in each segment as there would be more confidence in considering readings from the segment that has more readings as compared to considering the segments with less number of readings. 
\begin{figure}
\begin{center}
\includegraphics[scale=0.4]{lessReadings.png}
\caption{\small \sl Segment with less readings than minimum required}
\label{fig:lessReadings}
\end{center}
\end{figure}




This work was supported by AFRL contract FA 9750-10-C-0221, AFOSR contract 9550-10-C-0179 and CASE at Syracuse University.

\begin{thebibliography}{6}
\bibitem{dep1} M. Nathaniel Mead, Benefits of Sunlight: A Bright Spot for Human Health
\bibitem{dep2} Richard H. Grant and Gordon M. Heisler, Effect of Cloud Cover on UVB Exposure Under Tree Canopies: Will Climate Change Affect UVB Exposure? 
\bibitem{dep3}Richard H. Grant , Gordon M. Heisler and Wei Gao, Estimation of Pedestrian Level UV Exposure Under Trees
\bibitem{dep4} http://articles.mercola.com/sites/articles/archive/2012/03/26/maximizing-vitamin-d-exposure.aspx
\bibitem{dep5} Wong, C.F., Toomey, S., Fleming, R.A., Thomas, B.W., 1995, UV-B radiometry and dosimetry for solar measurements, Journal of the Health Physics Society,Vol. 68.
\bibitem{uvguardian}Jerrid Matthews, Farnoosh Javadi, Gauresh Rane, Jason Zheng, Giovanni Pau, Mario Gerla, Ultraviolet Guardian - Real Time Ultraviolet Monitoring
\end{thebibliography}


\end{document}




